\documentclass[12pt]{article}

\begin{document}
\title{The significance of cars in our society and their effects.}
\author{Ssempala Raymnd 15/U/12851/EVE}
\date{May, 21, 2017}
\maketitle

\section{Introduction}
Cars are a very important aspect in our lives today. Cars originate from the term vehicles which symbolizes a lot of mobile machines that transport people or cargo. There are different types of vehicles which include wagons, bicycles, motor vehicles (motorcycles, trucks, and buses), railed vehicles (trains, trams), watercrafts (ships, boats), aircrafts and spacecraft. But for our study we are going to concentrate motor vehicles specifically cars. Cars have been in existence since the 1672s up to date where modern cars are being manufactured. Modern self-driving cars are also being inverted.

\section{Background}
According to my research, basically I took a case study in Uganda where different people have different views about cars. There are a lot of merits and demerits about cars but they all depend on the different views people have.
On the other hand, there over 1billion cars in the world and a rough estimate of 3 million cars in Uganda according to people.  Different cars cost different prices and different people drive different cars for examples spacious, Subaru, range rovers, Mercedes Benz, mini cooper, Noah, wish etc. 

According to the research, there are problems faced by people as a result of the cars like accidents, unstable fuel consumption rates, expenses incurred during maintenances and many other problems caused to the environment like noise pollution and air pollution.

\section{Problem Statment}
This project will address the different challenges being faced by people which include accidents, costs, maintenances etc. and cars and more importantly the difficulties that arose as I was collecting the information.
In this section we are going to also consider the different hazards cars impact on the environment.

\section{Objectives}
\subsection{Main Objective}
To look at the significances of cars, the various types which include hatchbacks, sedans, MPVs, SUVs, crossovers, coupes etc. and in these different types there are examples Toyotas, Mercedes-Benz, Hondas, land rovers etc. we will also look at the various hazard, challenges and effects caused and faced by cars.

\section{Literature review}
Here we look at a number of different features a car contains. Most cars are propelled by and internal combustion engine, fueled by deflagration of gasoline or diesel. Cars are equipped with controls for driving, passenger comfort and safety. These controls include steering wheel, pedals for operating the brakes and control the car’s speed. And also cars are either manual transmission or automatic transmission.
	
Cars are typically fitted with multiple types of lights both inside the car and outside. They are also designed to carry occupants.
Cars also have negative effects to the society.

\section{Methodology}
Cars usually come with two transmission types. Manual transmission which consists of a speed pedal, brake pedal and clutch pedal. It also consists of a shift lever for changing gears. Before one shifts from one gear to the next, one has to first put his foot on the clutch then change the gear using a stick. 
Automatic transmission which consists of only two pedals i.e. the speed pedal and brake pedal. Here the driver does not need to change gears since the cars automatically does it for him.
According to this research different data collection tools were using but mostly the Open data kit was used. Where a server for information storage was developed using the Google Cloud Platform and an Aggregate server. The open data kit build was used to formulate forms that were used in the field.

\section{References}
1. Stein, Ralph (1967). The Automobile Book. Paul Hamlyn.
2. "motor car, n.". OED Online. Oxford University Press. September 2014. Retrieved 2014-09-29.
3. "auto-, comb. form2". OED Online. Oxford University Press. September 2014. Retrieved 2014-09-29.


\end{document}
